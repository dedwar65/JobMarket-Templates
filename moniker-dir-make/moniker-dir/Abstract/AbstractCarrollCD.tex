  Recent empirical evidence of heterogeneity in the rate of return (an important feature of the wealth accumulation process) for individuals provide motivation for an analogous assumption for households faced with a standard consumption-saving problem. In this way, one can test the theory regarding a meaningful relationship between stochastic returns and wealth inequality within a macroeconomic setting where the conditions for a stationary model distribution of wealth are satisfied. A uniform distribution of the rate of return across households is estimated such that empirical moments of wealth (net worth) measured in the 2004 survey of consumer finances SCF are matched particularly well. A lognormal distribution of returns is estimated, which is not only closer to the empirical distribution measured by Fagereng, Guiso, Malacrino, and Pistaferri (2020, but results in simulated wealth moments which better fit the empirical moments for net worth as well.
