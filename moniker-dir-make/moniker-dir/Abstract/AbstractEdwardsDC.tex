 Recent empirical evidence of heterogeneity in the rate of return (an important feature of the wealth accumulation process) for individuals provide motivation for an analogous assumption in a standard heterogeneous agent (HA) macroeconomic model. In the infinite horizon setting, a uniform distribution of the rate of return across households is estimated such that empirical moments of wealth (net worth) measured in the Survey of Consumer Finances are matched particularly well by their model counterparts. The fit of the model is explored after accommodating more realistic assumptions like life-cycle considerations, bequest motives, and portfolio choice as well. These findings suggest that heterogeneity in parameters which determine optimal consumption-saving behavior other than the time preference factor can generate meaningful wealth inequality. Factors which explain differences in returns across individuals could be used to endogenize heterogeneity in the rate of return, allowing for a more robust analysis of wealth inequality using macroeconomic models.
