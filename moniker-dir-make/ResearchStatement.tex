\RequirePackage{mymoniker}    % Replace CarrollCD in mymoniker.sty with your LastnameFirstNameMiddleName
\RequirePackage{dissertation} % required extra stuff
\documentclass{scrartcl}

\usepackage{color,hyperref}
% Uses hyperref to link DOI
\newcommand\doilink[1]{\href{http://dx.doi.org/#1}{#1}}
\newcommand\doi[1]{doi:\doilink{#1}}

\usepackage{dirtytalk}
% Packages
\usepackage[utf8]{inputenc}
\usepackage[backend=bibtex, style=authoryear]{biblatex} % You can change the style if you prefer a different citation style
\addbibresource{References.bib}


\begin{document}

%\begin{abstract}
%  % Change CarrollCD to your Moniker 
\providecommand{\mymoniker}{EdwardsDC}

% Leave the stuff below unchanged
\newwrite\mymonikerfile\immediate\openout\mymonikerfile=.mymoniker
\write\mymonikerfile{\mymoniker}


%\end{abstract}

\section{Ongoing and completed JMP research}

My job market paper, which is a work in progress, constructs a standard HA model of consumption-saving for households who are ex-post heterogeneous due to idiosyncratic labor income risk. 
Additionally, I allow for households to be heterogeneous ex-ante, as in the models by \cite{gkgv22} and \cite{cstw2017}. 
Most such models assume that the rate of return is identical across households, which very recent work by \cite{aflgdmlp20} has discovered evidence against. 

Preliminary results of a simulated method of moments estimation show that such rate of return heterogeneity is sufficient to match measurements of inequality from the Survey of Consumer Finances. 
This result holds both for the baseline model where households solve an infinite horizon problem, and for a realistically calibrated
life cycle version of the model. 

The next steps toward completing my job market paper will be to include both a bequest motive and choice of a risky asset for households.
Both are straightforward to accommodate and I have support from my department/advisor in getting this done. 
I will need to modify the structural estimation to allow for other moments of the model, like median wealth holdings by age, to 
match empirical counterparts of those moments in order to properly identify the estimated bequest parameter and distribution of returns.

After this, and carefully modifying the code to allow for portfolio choice, I will run robustness checks for this final version of the model,
which include matching moments from the different waves of the SCF wealth data, as well as for measures of wealth other than net worth (like liquid or financial wealth).


\subsection{Other directions for research}

My advisor and I meet regularly through the year to discuss my dissertation progress. 
In my attempt to find another research question to explore for the remaining chapters of my dissertation,
we have discussed the possibility of considering the relationship between trust and the rate of return, given the results
I have so far from my JMP.

\cite{jbpglg2016} uses data from the European Social Survey to establish an empirical relationship between trust levels measured in the survey and income data for the households. 
In particular, the paper finds that there is a \say{right amount of trust}. That is, the relationship between trust and income is humped shaped. 
An intermediate level of trust is associated with the maximal level of income. 

It would be interesting to see if I could use data to see if there is a similar relationship between individual trust levels and returns to wealth. 
In particular, does the hump-shaped relationship between trust and economic performance found in \cite{jbpglg2016} holds also for returns to wealth?
If so, perhaps this provides as adequate motivation to consider a model similar to the one from my JMP which allows for households to differ in there trust levels.

From there, I could use data on varying degrees of trust across individuals to calibrate the model and produce return heterogeneity through that channel. 
A nice result would be that this model, with heterogeneity in trust, also is able to generate significant inequality in the model's simulated distribution of wealth. 

\end{document}
