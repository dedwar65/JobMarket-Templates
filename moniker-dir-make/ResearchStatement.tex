\RequirePackage{mymoniker}    % Replace CarrollCD in mymoniker.sty with your LastnameFirstNameMiddleName
\RequirePackage{dissertation} % required extra stuff
\documentclass{scrartcl}

\usepackage{color,hyperref}
% Uses hyperref to link DOI
\newcommand\doilink[1]{\href{http://dx.doi.org/#1}{#1}}
\newcommand\doi[1]{doi:\doilink{#1}}

\usepackage{dirtytalk}
% Packages
\usepackage[utf8]{inputenc}
\usepackage[backend=bibtex, style=authoryear]{biblatex} % You can change the style if you prefer a different citation style
\addbibresource{References.bib}


\begin{document}

%\begin{abstract}
%  % Change CarrollCD to your Moniker 
\providecommand{\mymoniker}{EdwardsDC}

% Leave the stuff below unchanged
\newwrite\mymonikerfile\immediate\openout\mymonikerfile=.mymoniker
\write\mymonikerfile{\mymoniker}


%\end{abstract}

\section{Research interests and expertise}

My journey in the economics profession began with an interest in wealth inequality. In the first half of my time as a graduate student, I took as many theory courses as the Economics department offered; in fact, the writing sample attached is from my second-year paper on a theoretical approach to proposing alternative measures of wealth inequality.

In the fall of my third year in the program, I took an interest to heterogeneous agent (HA) macroeconomics for two main reasons. The first is that this area of research is a blend of mathematics (describing the solution to households' dynamic consumption-saving decisions) and computational methods. I share the perspective of the many researchers working on HA models: any model that hopes to be persuasive will require heterogeneity among the households; such models \textit{must} be solved numerically.

This pivot to computational macroeconomics has led me to hone my coding skills, specifically in Python (some of the many attractive features of Python for numerical computation are discussed at this \href{https://python-programming.quantecon.org/about_py.html}{link}). In fact, I am currently a research assistant for the \href{https://econ-ark.org}{Econ-ARK Project} headed by Prof. Christopher Carroll.\footnote{Chris is my principal advisor, and is one of my references.} The main set of tools provided by the project are in the \href{https://econ-ark.org/toolkit}{Heterogeneous Agents Resources and toolKit (HARK)}. Its primary goal is to provide \textit{reproducible} code to solve and extend these sorts of HA models.

\section{Ongoing and completed research}

I've mentioned the HARK project also because I am using those tools in my dissertation research. My job market paper, which is a work in progress, constructs a standard HA model of consumption-saving for households who are ex-post heterogeneous due to idiosyncratic labor income risk. Additionally, I allow for households to be heterogeneous ex-ante, as in the models by \cite{gkgv22} and \cite{cstw2017}. Most such models assume that the rate of return is identical across households, which very recent work by \cite{aflgdmlp20} has discovered evidence against. Using insights and elements from this paper, my paper examines how much wealth inequality would be produced if the only heterogeneity were in the rate of return. Preliminary results find that such rate of return heterogeneity is sufficient to match measurements of inequality from the Survey of Consumer Finances.

% Unlike the time preference factor for households, estimating differences in the rate of return to financial assets across households is possible, as this variable is directly observable. \cite{aflgdmlp20} conduct their own systematic analysis of return heterogeneity using 12 years of data from Norway's administrative tax records. The authors' findings reveal substantial differences in the average returns to assets for individuals (\textit{type dependence}), that this heterogeneity is found both within and across classes of assets with varying levels of risk, and that returns are positively correlated with wealth  (\textit{scale dependence}). Moreover, they further demonstrate that this discovery of heterogeneous returns exhibits significant persistence over time and are positively correlated across generations. Each of these findings provide not only motivation for the assumption of ex-ante heterogeneous rates of return in the buffer-stock savings model of households, but also provide a benchmark to compare the distribution of rates of return resulting from the estimation procedure aimed at best matching the empirical distribution of wealth.

\section{Project proposals for internship period}

Since rate of return heterogeneity is at the core of my research program, visiting the bank of Norway and learning more about how these data are constructed would allow for a major imporivement in the persuasiveness and credibility of my paper. I would hope also to gain a deeper understanding of the registry data and to explore whether there are other kinds of ex-ante heterogeneity which could or should be incorporated in a HA structural model.

%The authors of the paper offer potential explanations for why indivdiual returns are heterogeneous, but if it were possible for me to explore this in light of my calibration needs for my own paper this would be beneficial.

\subsection{Empirical estimates of a relationship between trust and the rate of return}

% The analysis presented by \cite{aflgdmlp20} can not only provide insights for alternative assumptions in HA macro models, but the dataset that the authors provide could also be used to understand important features of wealth accumulation for individuals, which may manifest themselves in the rate of return. In this paper, the authors note a significant idiosyncratic component for realized returns; when included in the statistical model, this individual fixed effect allows the model to better explain variations in returns.

The authors go on to mention \say{financial sophistication, the ability to process and use financial information, the ability to overcome inertia, and... the talent to manage and organize their businessess} as potential explanations for these persistent differences in returns across individuals. In the next phase of my research, I intend to explore another channel: the role of trust.

\cite{jbpglg2016} uses data from the European Social Survey to establish an empirical relationship between trust levels measured in the survey and income data for the households. In particular, the paper finds that there is a \say{right amount of trust}; that is, the relationship between trust and income is humped shaped. An intermediate level of trust is associated with the maximal level of income. 

Since the ESS data from the \cite{jbpglg2016} work includes Norway as one of the countries for which the intensity of trust levels is measured for individuals, my hope would be that it is possible to use either direct measures of trust, if any such are linked to the registry dataset, or proxies for trust estimated from other datasets. The next step would be to describe the empirical relationship between individual trust levels and registry-measured returns to wealth. In particular, it would be interesting to see if the hump-shaped relationship between trust and economic performance found in \cite{jbpglg2016} holds also for returns to wealth.

With these results, I could go back and examine whether heterogeneity in trust (which is pervasive in the United States) could potentialy explain a substantial portion of wealth inequality. 

\end{document}
